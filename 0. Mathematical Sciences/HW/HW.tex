\documentclass{article}
\usepackage[margin=1.1in]{geometry}
\usepackage{undertilde, amsmath, cancel}
\DeclareMathOperator{\tr}{tr}
\title{PHYS 7125 Homework 3}
\author{Wenqi He}
\begin{document}
\maketitle
\section{}
\[ \nabla_\lambda R_{\rho\sigma\mu\nu} + \nabla_\rho R_{\sigma\lambda\mu\nu} + \nabla_\sigma R_{\lambda\rho\mu\nu} = 0 \]
\[ g^{\mu\lambda} g^{\nu\sigma}\nabla_\lambda R_{\rho\sigma\mu\nu} + g^{\mu\lambda} g^{\nu\sigma} \nabla_\rho R_{\sigma\lambda\mu\nu} + g^{\mu\lambda} g^{\nu\sigma} \nabla_\sigma R_{\lambda\rho\mu\nu} = 0 \]
\[ \nabla_\lambda \Big(g^{\mu\lambda} g^{\nu\sigma} R_{\sigma\rho\nu\mu}\Big) - \nabla_\rho \Big(g^{\mu\lambda} g^{\nu\sigma}  R_{\lambda\sigma\mu\nu}\Big) + \nabla_\sigma \Big(g^{\mu\lambda} g^{\nu\sigma} R_{\lambda\rho\mu\nu}\Big) = 0 \]
\[ \nabla_\lambda \Big(g^{\mu\lambda} R_{\rho\mu}\Big) - \nabla_\rho \Big( g^{\nu\sigma}  R_{\sigma\nu}\Big) + \nabla_\sigma \Big( g^{\nu\sigma} R_{\rho\nu}\Big) = 0 \]
\[ \nabla_\lambda \Big(g^{\mu\lambda} R_{\rho\mu}\Big) - \nabla_\rho R+ \nabla_\lambda\Big( g^{\mu\lambda} R_{\rho\mu}\Big) = 0 \]
\[ \nabla_\lambda \Big(g^{\mu\lambda} R_{\rho\mu}\Big) - \nabla_\lambda \Big( \frac{1}{2}\delta_\rho^\lambda R\Big)  = 0 \]
\[ \nabla_\lambda \Big(g^{\mu\lambda} R_{\rho\mu}\Big) - \nabla_\lambda \Big (\frac{1}{2} g^{\mu\lambda}g_{\rho\mu} R\Big)  = 0 \]
\[  \nabla_\lambda  \Big[ g^{\mu\lambda} \Big(  R_{\rho\mu} - \frac{1}{2} g_{\rho\mu} R\Big) \Big]  = 0 \]
\[  \nabla_\lambda  \Big( g^{\mu\lambda} G_{\rho\mu} \Big)  = 0 \]
\[  \nabla_\lambda  G^\lambda{}_\rho  = 0 \]

\section{}
The only non-vanishing components of the metric are 
\[ g_{\psi\psi} = 1, \quad  g_{\theta\theta}= \sin^2\psi, \quad g_{\phi\phi} = \sin^2\psi \sin^2\theta\]
Since the matrix is diagonal, the corresponding components of the inverse metric are simply
\[ g^{\psi\psi} = 1, \quad  g^{\theta\theta} = 1/\sin^2\psi, \quad g^{\phi\phi} = 1/\sin^2\psi \sin^2\theta\]
The only non-vanishing first derivatives of the metric components are
\[ g_{\theta\theta,\psi} = 2\sin\psi\cos\psi, \quad g_{\phi\phi,\psi} = 2\sin\psi\cos\psi\sin^2\theta,
	\quad g_{\phi\phi,\theta} = 2\sin^2\psi \sin\theta\cos\theta \]
\subsection*{a}
The only non-vanishing Christoffel symbols are
\[ \Gamma^\theta_{\theta\psi} = \Gamma^\theta_{\psi\theta} = \frac{1}{2}g^{\theta\theta}g_{\theta\theta,\psi}
	= \frac{2\sin\psi\cos\psi}{2\sin^2\psi} = \cot\psi \]
\[ \Gamma^\psi_{\theta\theta} = -\frac{1}{2}g^{\psi\psi}g_{\theta\theta,\psi} = -\sin\psi\cos\psi \]
\[ \Gamma^\phi_{\phi\psi} = \Gamma^\phi_{\psi\phi} = \frac{1}{2}g^{\phi\phi}g_{\phi\phi,\psi}
	= \frac{2\sin\psi\cos\psi\sin^2\theta}{2\sin^2\psi\sin^2\theta} = \cot\psi \]
\[ \Gamma^\psi_{\phi\phi} = -\frac{1}{2}g^{\psi\psi}g_{\phi\phi,\psi} = -\sin\psi\cos\psi\sin^2\theta\]
\[ \Gamma^\phi_{\phi\theta} =  \Gamma^\phi_{\theta\phi} = \frac{1}{2}g^{\phi\phi}g_{\phi\phi,\theta}
	= \frac{2\sin^2\psi\sin\theta\cos\theta}{2\sin^2\psi\sin^2\theta} = \cot\theta\]
\[ \Gamma^\theta_{\phi\phi} = -\frac{1}{2}g^{\theta\theta}g_{\phi\phi,\theta}
	= -\frac{2\sin^2\psi\sin\theta\cos\theta}{2\sin^2\psi} = -\sin\theta\cos\theta \]
\subsection*{b}
There are $\dfrac{1}{12}\cdot3^2\cdot(3^2-1) = 6$ independent components (others can be obtained by symmetry) 
\begin{align*}
R_{\psi\theta\psi\theta} &= g_{\psi\psi}R^{\psi}_{\theta\psi\theta} = \sin^2\psi-\cos^2\psi - 0 + 0 - (-\cos^2\psi) = \sin^2\psi \\
R_{\psi\theta\psi\phi} &= g_{\psi\psi}R^{\psi}_{\theta\psi\phi} = 0 - 0 + 0 - 0 = 0 \\
R_{\psi\theta\theta\phi} &= g_{\psi\psi}R^{\psi}_{\theta\theta\phi} = 0 - 0 + 0 - 0 = 0 \\
R_{\psi\phi\psi\phi} &= g_{\psi\psi}R^{\psi}_{\phi\psi\phi} =  (\sin^2\psi-\cos^2\psi)\sin^2\theta - 0 + 0 - (-\cos^2\psi\sin^2\theta) = \sin^2\psi\sin^2\theta \\
R_{\psi\phi\theta\phi} &= g_{\psi\psi}R^{\psi}_{\phi\theta\phi} = -2\sin\psi\cos\psi\sin\theta\cos\theta - 0 + \sin\psi\cos\psi\sin\theta\cos\theta - (-\sin\psi\cos\psi\sin\theta\cos\theta) = 0 \\
R_{\theta\phi\theta\phi} &= g_{\theta\theta}R^{\theta}_{\phi\theta\phi} = \sin^2\psi\Big[\sin^2\theta -\cos^2\theta - 0 + (-\cos^2\psi\sin^2\theta) - (-\cos^2\theta)\Big] = \sin^4\psi\sin^2\theta \\\\
R_{\psi\psi} &= g^{\theta\theta}R_{\theta\psi\theta\psi} + g^{\phi\phi}R_{\phi\psi\phi\psi} = 2\\
R_{\psi\theta} &= g^{\phi\phi}R_{\phi\psi\phi\theta} = 0\\
R_{\psi\phi} &= g^{\theta\theta}R_{\theta\psi\theta\phi} = 0 \\
R_{\theta\theta} &= g^{\psi\psi}R_{\psi\theta\psi\theta} + g^{\phi\phi}R_{\phi\theta\phi\theta} = \sin^2\psi + \sin^2\psi = 2\sin^2\psi\\
R_{\theta\phi} &= g^{\psi\psi}R_{\psi\theta\psi\phi} = 0\\
R_{\phi\phi} &= g^{\psi\psi}R_{\psi\phi\psi\phi} + g^{\theta\theta}R_{\theta\phi\theta\phi} = \sin^2\psi\sin^2\theta + \sin^2\psi\sin^2\theta = 2\sin^2\psi\sin^2\theta\\\\
R &= g^{\psi\psi}R_{\psi\psi} + g^{\theta\theta}R_{\theta\theta} + g^{\phi\phi}R_{\phi\phi} = 2 + 2 + 2 = 6
\end{align*}
\section{}
Proposed topic: alternative theories of gravity, specifically, Einstein-Cartan and Brans-Dicke

\end{document}