\documentclass{article}
\usepackage[margin=1.1in]{geometry}
\usepackage{undertilde, amsmath, amsfonts, commath, cancel}
\title{PHYS 7125 Homework 4}
\author{Wenqi He}
\begin{document}
\maketitle
\section{}
For a timelike vector $t^\mu$ at any point $p$ it is always possible to construct an orthonormal frame $\utilde{\mathbf{e}}_i^\mu$  where $\utilde{\mathbf{e}}_0^\mu= t^\mu$. (Without loss of generality, $t^\mu$ can be assumed to have unit length (-1). In the general case, the results only differ by a positive factor.) In such coordinates, $t^\mu = (1,0,0,0)$, the metric is locally $\eta_{\mu\nu}$, and the components of the electromagnetic energy-momentum tensor is
\[ T_{\mu\nu} = \frac{1}{\mu_0}\Big[ F_\mu{}^\alpha F_{\nu\alpha} - \frac{1}{4}\eta_{\mu\nu}F_{\alpha\beta}F^{\alpha\beta} \Big] \]
Considering $t^\mu$ is only non-zero in its time component and $T_{\mu\nu}$ is symmetric, we only need to compute
\begin{align*}
T_{0\nu} = \frac{1}{\mu_0}\Big[ F_0{}^\alpha F_{\nu\alpha} - \frac{1}{4}\eta_{0\nu}F_{\alpha\beta}F^{\alpha\beta} \Big] =: A+ B 
\end{align*}
The first term evaluates to: (Define $\mathbf{S} := \frac{1}{\mu_0}\mathbf{E} \times \mathbf{B}$)
\[
	A = \frac{1}{\mu_0} \begin{pmatrix}
		0 & E_x/c & E_y/c & E_z/c \\
		-E_x/c & 0 & -B_z & B_y \\
		-E_y/c & B_z & 0 & -B_x \\
		-E_z/c & -B_y & B_x & 0
	\end{pmatrix} \begin{pmatrix}
		0 \\ E_x/c \\ E_y/c \\ E_z/c
	\end{pmatrix} = \begin{pmatrix}
		\epsilon_0 E^2 \\
		-\frac{1}{\mu_0}(E_yB_z - E_zB_y)/c \\
		-\frac{1}{\mu_0}(E_zB_x - E_xB_z)/c \\
		-\frac{1}{\mu_0}(E_xB_y - E_yB_x)/c \\
	\end{pmatrix} = \begin{pmatrix}
		\epsilon_0 E^2 \\
		-S_x/c \\
		-S_y/c \\
		-S_z/c \\
	\end{pmatrix}
\]
The second term:
\[
B = -\frac{1}{4\mu_0} (-2E^2/c^2 + 2B^2)\begin{pmatrix}
	-1\\ 0 \\ 0\\ 0
\end{pmatrix} = \begin{pmatrix}
	  \frac{1}{2}\Big(-\epsilon_0E^2 + \frac{1}{\mu_0}B^2 \Big)\\ 0 \\0 \\ 0
	\end{pmatrix}
\quad\Rightarrow\quad T_{0\nu} = \begin{pmatrix}
		\frac{1}{2} \Big(\epsilon_0 E^2 + \frac{1}{\mu_0}B^2 \Big) \\
		-S_x/c \\
		-S_y/c \\
		-S_z/c \\
	\end{pmatrix}
 \]
In matrix form, the relevant components of the energy-momentum tensor are:
\[
	T_{\mu\nu} = \begin{pmatrix}
		\frac{1}{2}\Big(\epsilon_0E^2 + \frac{1}{\mu_0}B^2\Big) & -S_x/c & -S_y/c & -S_z/c\\
		-S_x/c & \cdots & \cdots & \cdots \\
		-S_y/c & \cdots & \cdots & \cdots \\
		-S_z/c & \cdots & \cdots & \cdots \\
	\end{pmatrix}
\]
(i)
The weak energy condition is obviously satisfied:
\[ T_{\mu\nu}t^\mu t^\nu = T_{00} = \frac{1}{2}\Big(\epsilon_0E^2 + \frac{1}{\mu_0}B^2\Big) \geq 0 \]
(ii)
\[ T_{\mu\nu}t^\mu = T_{0\nu} = \Bigg(\frac{1}{2}\Big(\epsilon_0E^2 + \frac{1}{\mu_0}B^2\Big), -S_x/c, -S_y/c, -S_z/c\Bigg)\]
\[ T^\nu{}_{\alpha}t^\alpha = T^\nu{}_{0} = g^{\nu\nu} \cdot T_{\nu0} =  \Bigg(-\frac{1}{2}\Big(\epsilon_0E^2 + \frac{1}{\mu_0}B^2\Big), -S_x/c, -S_y/c, -S_z/c\Bigg)\]
Using Lagrange's identity for cross products,
\begin{align*}
(T_{\mu\nu}t^\mu) (T^\nu{}_{\alpha}t^\alpha) &=  -\frac{1}{4}\Big(\epsilon_0E^2 + \frac{1}{\mu_0}B^2\Big)^2 + \frac{\norm{\mathbf{S}}^2}{c^2} \\
&=  -\frac{1}{4}\Big(\epsilon_0E^2 + \frac{1}{\mu_0}B^2\Big)^2 + \frac{\epsilon_0}{\mu_0} \norm{\mathbf{E}\times\mathbf{B}}^2 \\
&=  -\frac{1}{4}\Big(\epsilon_0E^2 + \frac{1}{\mu_0}B^2\Big)^2 + \frac{\epsilon_0}{\mu_0} \Big(E^2B^2 - (\mathbf{E}\cdot\mathbf{B})^2\Big) \\
&= -\frac{1}{4}\Big(\epsilon_0E^2 - \frac{1}{\mu_0}B^2\Big)^2 - \frac{\epsilon_0}{\mu_0}(\mathbf{E}\cdot\mathbf{B})^2
\leq 0
\end{align*}
Thus the dominant energy condition is also satisfied.
\section{}
\subsection*{a}
The only non-vanishing components of $\Gamma^{\mu}_{0\nu}$ are
\[ \Gamma^{0}_{0i} = \frac{Mx^i}{(1-2M/r)r^3}, \quad \Gamma^{i}_{00} = \frac{Mx^i}{(1+2M/r)r^3} \quad\text{  ($i = 1,2,3$) }\]
The geodesic equation can be simplified as
\begin{align*}
\frac{dp_0}{d\lambda} &= \sum_i \Bigg( \frac{Mx^i}{(1-2M/r)r^3}p_0 p^i + \frac{Mx^i}{(1+2M/r)r^3}p^0 p_i \Bigg)\\
&= \sum_i \Bigg( g_{00}\frac{Mx^i}{(1-2M/r)r^3} + g_{ii}\frac{Mx^i}{(1+2M/r)r^3} \Bigg)p^0 p^i\\
&= \sum_i \Bigg( -\frac{Mx^i}{r^3} + \frac{Mx^i}{r^3} \Bigg)p^0 p^i = 0
\end{align*}
\subsection*{b}
No. $p^0 = g^{0\nu}p_\nu = g^{00}p_0 = -(1-2M/r)^{-1}p_0$, which is not constant unless $r$ is constant.
\subsection*{c}
For the atom at rest on the surface of the sun, $dx^i = 0$, $r = R$,
\[ d\tau^2 = -ds^2 = (1-2M/R)dt^2 \quad\Rightarrow\quad u^0 = \frac{dt}{d\tau} = \frac{1}{\sqrt{1-2M/R}} \]
\subsection*{d}
For both the atom and the distant observer, $dx^i = 0$,
\[ d\tau^2  = -ds^2 = (1-2M/r)dt^2 \]
\[ \Rightarrow\quad u^\mu = \frac{dx^\mu}{d\tau} = (dt/d\tau,0,0,0)  = ((1-2M/r)^{-1/2}, 0, 0,0) \]
The photon energy observed at both locations can be expressed as
\[ E = g_{\mu\nu}p^\mu u^\nu = (1-2M/r)^{-1/2} g_{\mu\nu}p^\mu K^\nu \]
where $K^\nu = (1,0,0,0)$ is a Killing vector as the metric has no time dependence; therefore $g_{\mu\nu}p^\mu K^\nu$ is conserved along the photon's world line, which is a geodesic as stated in the problem. Then,
\[ \frac{\lambda_r}{\lambda_e} = \frac{h c /\lambda_e}{h c/\lambda_r} = \frac{E_e}{E_r} = \frac{(1-2M/R)^{-1/2}}{\lim_{r\rightarrow\infty} (1-\cancel{2M/r})^{-1/2}} = 1 + \frac{M}{R} + \mathcal{O}\left(\frac{M^2}{R^2}\right)\]
\[ z = \frac{\lambda_r - \lambda_e}{\lambda_e} = \frac{\lambda_r}{\lambda_e} - 1 = \frac{M}{R} + \mathcal{O}\left(\frac{M^2}{R^2}\right) \]


\end{document}