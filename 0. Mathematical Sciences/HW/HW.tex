\documentclass{article}
\usepackage[margin=1.1in]{geometry}
\usepackage{undertilde, amsmath, amsfonts, commath, cancel}
\title{PHYS 7125 Homework 4}
\author{Wenqi He}
\begin{document}
\maketitle
\section{}
For any time-like vector $t^\mu$ we can always choose a local coordinate system such that $g_{\mu\nu} = \eta_{\mu\nu}$ and $t^\mu = (1,0,0,0)$. The electromagnetic energy-momentum tensor expressed in such coordinates is
\[
	T_{\mu\nu} = \begin{pmatrix}
		\frac{1}{2}(\epsilon_0E^2 + \frac{1}{\mu_0}B^2) & -S_x/c & -S_y/c & -S_z/c\\
		-S_x/c & \cdots & \cdots & \cdots \\
		-S_y/c & \cdots & \cdots & \cdots \\
		-S_z/c & \cdots & \cdots & \cdots \\
	\end{pmatrix}
\]
where \[\mathbf{S} = \frac{1}{\mu_0}\mathbf{E} \times \mathbf{B}\]
(i)
First, the weak energy condition is satisfied:
\[ T_{\mu\nu}t^\nu t^\nu = \frac{1}{2}\Big(\epsilon_0E^2 + \frac{1}{\mu_0}B^2\Big) \geq 0 \]
(ii)
\[ T_{\mu\nu}t^\mu = \Big(\frac{1}{2}(\epsilon_0E^2 + \frac{1}{\mu_0}B^2), -S_x/c, -S_y/c, -S_z/c\Big)\]
Raising one index,
\[ T^\nu{}_{\alpha}t^\alpha = \Big(-\frac{1}{2}\Big(\epsilon_0E^2 + \frac{1}{\mu_0}B^2\Big), -S_x/c, -S_y/c, -S_z/c\Big)\]
Using Lagrange's identity for cross products,
\begin{align*}
(T_{\mu\nu}t^\mu) (T^\nu{}_{\alpha}t^\alpha) &= -\frac{1}{4}\Big(\epsilon_0^2E^4 + \frac{1}{\mu_0^2}B^4 + 2\frac{\epsilon_0}{\mu_0}E^2B^2\Big) + \frac{\norm{\mathbf{S}}^2}{c^2} \\
&= -\frac{1}{4}\Big(\epsilon_0^2E^4 + \frac{1}{\mu_0^2}B^4 + 2\frac{\epsilon_0}{\mu_0}E^2B^2\Big) + \frac{\epsilon_0}{\mu_0} (E^2B^2 - (\mathbf{E}\cdot\mathbf{B})^2) \\
&= -\frac{1}{4}\Big(\epsilon_0E^2 - \frac{1}{\mu_0}B^2\Big)^2 - \frac{\epsilon_0}{\mu_0}(\mathbf{E}\cdot\mathbf{B})^2
\leq 0 \\
\end{align*}
Thus the dominant energy condition is satisfied.
\section{}
\subsection*{a}
Since the metric is diagonal, indices can be lowered by a simple multiplication
\[ V_\mu = \sum_\nu V^\nu g_{\mu\nu} = V^\mu \cdot g_{\mu\mu}\]
The non-vanishing components of $\Gamma^{\mu}_{0\nu}$ are ($i = 1,2,3$)
\[ \Gamma^{0}_{0i} = \frac{Mx^i}{(1-2M/r)r^3}, \quad \Gamma^{i}_{00} = \frac{Mx^i}{(1+2M/r)r^3} \]
The geodesic equation can be simplified as
\begin{align*}
\frac{dp_0}{d\lambda} &= \sum_i \Big[ \frac{Mx^i}{(1-2M/r)r^3}p_0 p^i + \frac{Mx^i}{(1+2M/r)r^3}p^0 p_i \Big]\\
&= \sum_i \Big[ g_{00}\frac{Mx^i}{(1-2M/r)r^3}p^0 p^i + g_{ii}\frac{Mx^i}{(1+2M/r)r^3}p^0 p^i \Big]\\
&= \sum_i \Big[ -\frac{Mx^i}{r^3}p^0 p^i + \frac{Mx^i}{r^3}p^0 p^i \Big] = 0
\end{align*}
\subsection*{b}
The geodesic equation for $p^0$ is
\[ \frac{dp^0}{d\lambda} = \Gamma^0_{\mu\nu}p^\mu p^\nu \]
The only non-vanishing $\Gamma^0_{\mu\nu}$ are ($i = 1,2,3$)
\[ \Gamma^{0}_{0i} = \Gamma^{0}_{i0} = \frac{Mx^i}{(1-2M/r)r^3} \]
The geodesic equation shows that $dp^0/d\lambda$ is generally non-zero:
\[ \frac{dp^0}{d\lambda} = \sum_i \Big[ \frac{2Mx^i}{(1-2M/r)r^3}p^0 p^i \Big] \not\equiv 0\]
\subsection*{c}
Since the atom is at rest, $u^i = 0$. The geodesic equation is
\[ \frac{du^0}{d\lambda} = \sum_i \Big[ \frac{2Mx^i}{(1-2M/r)r^3}u^0 u^i \Big] = 0\]
Therefore $u^0 = dt/d\lambda$ could be any constant, which means that $\lambda = at + b$.
\subsection*{d}
Since both the atom and the oberver are at rest, $dx^i = 0$,
\[ ds^2 = -(1-2M/r)dt^2 = -d\tau^2 \quad\Rightarrow\quad d\tau  = (1-2M/r)^{1/2}dt\]
On the surface of the sun $d\tau_e  = (1-2M/R)^{1/2}dt$ and far away from the sun $d\tau_r \approx dt$, so the time dilation is $d\tau_e / d\tau_r = (1-2M/R)^{1/2}$. And since the speed of light is constant everywhere, 
\[ \frac{\lambda_e}{\lambda_r} = \frac{c\Delta\tau_e}{c\Delta\tau_r} = (1-2M/R)^{1/2} = 1 - \frac{M}{R} + \mathcal{O}\left((M/R)^2\right)\]
\[ \frac{\lambda_r - \lambda_e}{\lambda_r} = \frac{M}{R} + \mathcal{O}\left((M/R)^2\right) \]


\end{document}