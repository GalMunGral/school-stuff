\documentclass{article}
\usepackage[margin=1.1in]{geometry}
\usepackage{undertilde, amsmath, amsfonts, commath, cancel}
\usepackage{accents}

\newcommand{\ut}[1]{\underaccent{\tilde}{#1}}

\title{PHYS 7125 Homework 5}
\author{Wenqi He}
\begin{document}
\maketitle
\section{}
The total proper time along a curve $\gamma$ is
\begin{align*}
\tau_{total} &= \int_\gamma d\tau = \int_\gamma \sqrt{-ds^2} \\
&= \int_\gamma \sqrt{-\Big(\frac{2M}{r} - 1\Big)dt^2 + \Big(\frac{2M}{r} -1 \Big)^{-1} dr^2 - r^2d\Omega^2}
\end{align*}
Inside the event horizon, $2GM/r > 1$, the first and thrid term in the square root are negative, therefore
\begin{align*}
\tau_{total} &< \int^{2M}_0 \sqrt{\Big(\frac{2M}{r} -1 \Big)^{-1}} dr  \\
&= \Bigg[-\sqrt{2Mr - r^2} - M \tan^{-1} \left(\frac{M-r}{\sqrt{2Mr - r^2}}\right) \Bigg] \Bigg\rvert^{2M}_0 \\
&= M \tan^{-1} \left(\frac{M-r}{\sqrt{2Mr - r^2}}\right) \Bigg\rvert_{2M}^0 = \pi M
\end{align*}
\section{}
\subsection*{a}
[insert diagram here]
\subsection*{b}
Yes, because the observer has to be massive. Furthermore, if the worldline is lightlike, then the light signal in (c) will never reach the falling observer because the two worldlines would be parallel straight lines in the Kruskal diagram.
\subsection*{c}
At constant $r = R$,
\[  T = \frac{1}{2}\sinh\left( \frac{t}{4GM} \right), \quad X = \frac{1}{2}\cosh\left( \frac{t}{4GM} \right)
	,\quad 4X^2 - 4T^2 = 1 \]
At $t = 0$ on this worldline,
\[ T = \frac{1}{2}\sinh0 = 0, \quad X = \frac{1}{2}\cosh0 = \frac{1}{2} \]
At the singularity,
\[ T = \cosh\left( \frac{t}{4GM} \right),\quad X = \sinh\left( \frac{t}{4GM} \right), \quad T^2 - X^2 = 1 \]
A timelike straight line passing through $(1/2, 0)$ can be expressed as 
\[ X = kT + \frac{1}{2}\]
where $-1 < k< 1$. When the first observer reaches singularity,
\[ (1- k^2)T^2 - kT - \frac{5}{4} = 0 \quad\Rightarrow\quad \boxed{T_s = \frac{k + \sqrt{5-4k^2}}{2(1-k^2)},\quad
	X_s = \frac{k^2 + k\sqrt{5-4k^2}}{2(1-k^2)} + \frac{1}{2}}\]
To reach this critical point, the photons emitted by the second observer must follow
\[ X  = - T + T_s + X_s \]
When the photon's worldline in the past intersects with that of the second observer,
\[ \cancel{4T^2} + 4(T_s + X_s)^2 -8(T_s+X_s)T - \cancel{4T^2} = 1\]
\[ \Rightarrow T = \frac{1}{2}\sinh\left( \frac{t}{4GM} \right)  = \frac{4(T_s+X_s)^2 - 1}{8(T_s + X_s)}
	,\quad \boxed{t = 4GM\sinh^{-1}\left(\frac{4(T_s+X_s)^2 - 1}{4(T_s + X_s)}\right)} \]
which is the latest Schwarzschild time that the second observer should send the signal. When the worldline of the first observer is a vertical line, $k=0$, $t \approx 4.70GM$
\section{}
A perfect fluid is incompressible, therefore $\nabla_\mu \rho = 0$. From the continuity equation,
\[ \nabla_\mu(\rho u^\mu) =  \rho \nabla_\mu u^\mu = 0 \quad\Rightarrow\quad \nabla_\mu u^\mu = 0\]
Since the covariant divergence of stress-energy tensor must vanish,
\begin{align*}
\nabla_\mu T^\mu{}_\nu &= (\nabla_\mu p) u^\mu u_\nu + (p+\rho) \cancel{(\nabla_\mu u^\mu)} u_\nu + (p+\rho) u^\mu (\nabla_\mu u_\nu) + \delta^\mu_\nu \nabla_\mu p \\
&=  u_\nu (\nabla_{\ut{u}} p)  + (p+\rho) \nabla_{\ut{u}} u_\nu + \nabla_\nu p \\
&= 0
\end{align*}
If the free index $\nu$ is dropped,
\[ (p + \rho)\nabla_{\ut{u}}\ut{u} = - \nabla p - \ut{u}\nabla_{\ut{u}}p \]

\end{document}