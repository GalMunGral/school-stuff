\documentclass[12pt]{article}
\usepackage[margin=1.2in]{geometry}
\usepackage{cite, undertilde, amsmath, amsfonts, commath, cancel, graphicx, cases, siunitx}
\usepackage[hyphens]{url}
\usepackage{breakurl}
\usepackage[breaklinks]{hyperref}
\usepackage{accents}

\newcommand{\ut}[1]{\underaccent{\tilde}{#1}}

\title{\textsc{\Large{PHYS-7125 Term Paper} \\ \LARGE{Overview of Einstein-Cartan Theory of Gravitation}}}
\author{Wenqi He}
\begin{document}
\sloppy
\maketitle
\section{Mathematical Considerations}
\subsection{Cartan's torsion tensor}
On a manifold with connection coefficients $\Gamma^k_{ij}$, Cartan's torsion tensor is defined as
\[ S_{ij}{}^k := \Gamma^k_{[ij]} = \frac{1}{2}(\Gamma^k_{ij} - \Gamma^k_{ji})\]
which measures the extent to which infinitesimal parallelograms fail to close. A vanishing torsion, as the fundamental theorem of Riemannian geometry states \cite{nlab:fundamental_theorem_of_riemannian_geometry}, leads to a unique metric-compatible connection on a (pseudo-)Riemanian manifold, which is the familiar Levi-Civita connection:
\[ \Gamma^l_{ij} = \{\substack{l\\ij}\} = \frac{1}{2}g^{lk}\Big(\partial_j g_{ik} + \partial_i g_{jk} - \partial_k g_{ij} \Big)\]
Here $\Gamma$ denotes connection coefficients, and $\{\}$ is the historic notation of the Christoffel symbols as appeared in Christoffel's 1869 paper and Einstein's 1915-1916 papers. From a purely mathematical perspective, it is not immediately obvious whether the torsion-free constraint is needed for a successful theory of gravitation. In fact, Einstein originally introduced Christoffel symbols not based on the aforementioned uniqueness theorem, but rather on his postulate that any quantity describing gravitation should depend solely on the metric and its derivatives \cite{einstein1916foundation}. However, even Einstein himself considered this assumption unnecessary in his later years. \footnote{...the essential achievement of general relativity, namely to overcome ``rigid" space (i.e. the inertial frame), is only indirectly connected with the introduction of a Riemannian metric. The directly relevant conceptual element is the ``displacement field" ($\Gamma$), which expresses the infinitesimal displacement of vectors. ... This makes it possible to construct tensors by differentiation and hence to dispense with the introduction of ``rigid" space (the inertial frame). In the face of this, it seems to be of secondary importance in some sense that some particular $\Gamma$ field can be deduced from a Riemannian metric... (Einstein, 1955)}
\subsection{Riemann-Cartan spacetime $U_4$}
If torsion is included, the spacetime geometry is non-Riemannian. In terms of connection coefficients, extra terms are added to the Christoffel symbol:
\[ \Gamma^l_{ij} = \{\substack{l\\ij}\} + S_{ij}{}^k - S_j{}^k{}_i + S^k{}_{ij} =: \{\substack{l\\ij}\} - K_{ij}{}^k \]
The Riemann curvature tensor can be defined as usual to be the measurement of changes in vector components after the vector is parallelly transported around an infinitesimal area back to its starting point:
\[ R^l{}_{kij} = 2\partial_{[i}\Gamma^l_{j]k} + 2\Gamma^l_{[i|m}\Gamma^m_{|j]k} \] 
Since the definition of Riemann tensor does not depend on the particular form of connection, it is still antisymmetric in the first two and last two indices and still satisfies the Bianchi identity, although this is in general not the case for other identities. By symmetry, the Ricci tensor $R_{ij}$ is still the only meaningful contraction of Riemann tensor, and the Einstein tensor $G_{ij}$ can also be defined in the same way as in general relativity. However, due to $\Gamma$ being asymmetric, $R_{ij}$ and $G_{ij}$ will be asymmetric in general. Riemann-Cartan spacetime $U_4$ fits in the hierarchy of manifolds as follows:
\[ (L_4, g) \xrightarrow{\nabla g = 0} U_4 \xrightarrow{S=0} V_4 \xrightarrow{R = 0} R_4\]
In other words, the most general manifold equipped with a metric-preserving connection is the Riemann-Cartan $U_4$. When torsion vanishes, it reduces to the pseudo-Riemannian $V_4$ in general relativity. If in addition curvature vanishes, the manifold becomes Minkowskian $R_4$.

\section{Field Equations of Einstein-Cartan Theory}
\subsection{Motivation}
From the perspective of particle physics, all elmentary particles can be classified by the representations of the Poincar\'e group and be labeled by mass $m$ and spin $s$. $m$ is associated with the translational part of Poincar\'e group and $s$ with the rotational part \cite{hehl1976general}. With spin being such a fundamental notion that is distinct from mass, it is reasonble to hypothesize that spin angular momentum is also a source of a gravitational field. That is, spin can be supposed to be coupled to some geometrical quantity that characterizes the rotational degrees of freedom of spacetime, analogous to how energy-momentum is coupled to metric.
\subsection{Canonical tensors}
Suppose a matter field $\psi(x^a)$ is distributed over spacetime and its action is
\[ S_{matter} = \int \mathfrak{L}(\psi,\partial\psi, g,\partial g, S) d^4x \]
From Noether's theorem, it can be shown that the canonical spin angular momentum tensor, defined as the Noether current for rotations (Lorentz transformations), can be expressed as the variational derivative of the Lagrangian density with respect to contortion, or the non-Christoffel part of the connection \cite{hehl1976general}:
\[\tau_k{}^{ji} := \frac{1}{\sqrt{-g}}\frac{\delta\mathfrak{L}}{\delta K_{ij}{}^k}\]
Additionally, with the following definitions
\[ \sigma^{ij} := \frac{2}{\sqrt{-g}}\frac{\delta\mathfrak{L}}{\delta g_{ij}},
\quad \mu_{k}{}^{ji} := \frac{1}{\sqrt{-g}}\frac{\delta\mathfrak{L}}{\delta S_{ij}{}^k},
\quad \stackrel{*}{\nabla}_k := \nabla_k + 2S_{kl}{}^l\]
the canonical energy-momentum tensor, or the Noether current for translations, is \cite{hehl1976general}:
 \[ \Sigma^{ij} := \sigma^{ij} - \stackrel{*}{\nabla}_k\mu^{ijk}  = \sigma^{ij} + \stackrel{*}{\nabla}_k(\tau^{ijk} - \tau^{jki} + \tau^{kij})\]
where $\sigma^{ij}$, the variational derivative of the Lagrangian density with respect to metric, is the familiar Hilbert stress-energy tensor, which alone is equivalent to the canonical energy-momentum tensor within the framework of general relativity. In Einstein-Cartan theory, however, the additional divergence term must be included to compensate for a non-zero torsion.
\subsection{Variational principles}
The field equations can be derived as always by doing a variation of the full action with respect to each independent variable, in this case $g_{ij}$ and $S_{ij}{}^k$. 
The most natural choice for the Lagrangian density of gravitational field is still the Einstein–Hilbert one:
\[\mathfrak{L}_{gravity} = \frac{1}{2\kappa}\mathfrak{R} = \frac{1}{2\kappa}\sqrt{-g} R\]
where $\kappa = 8\pi G$. The total action is
\[ S = S_{matter} + S_{gravity} = \int \Big[ \mathfrak{L}(\psi,\partial\psi, g,\partial g, S) +  \frac{1}{2\kappa} \sqrt{-g}R  \Big]d^4x\]
Varying the action with respect to $g_{ij}$ and $S_{ij}{}^k$ gives:
\begin{numcases}{}
	- \frac{\delta(\sqrt{-g}R)}{\delta g_{ij}} = \kappa \sqrt{-g} \left( \frac{2}{\sqrt{-g}}\frac{\delta{\mathfrak{L}}}{\delta g_{ij}} \right) = \kappa \sqrt{-g} \sigma^{ij} \label{a} \\
	- \frac{\delta(\sqrt{-g}R)}{\delta S_{ij}{}^k} = 2\kappa \sqrt{-g} \left( \frac{1}{\sqrt{-g}}\frac{\delta{\mathfrak{L}}}{\delta S_{ij}{}^k} \right) = 2\kappa \sqrt{-g} \mu_k{}^{ji}
\end{numcases}
Using the definition of $\Sigma^{ij}$ and equation (2), equation (1) can be written as:
\begin{align*} 
- \frac{\delta(\sqrt{-g}R)}{\delta g_{ij}}  &= \kappa \sqrt{-g} \left( \Sigma^{ij} + \stackrel{*}{\nabla}_k\mu^{ijk}\right)
= \kappa \sqrt{-g} \left[ \Sigma^{ij} + \stackrel{*}{\nabla}_k \left(\frac{g^{li}}{2\kappa\sqrt{-g}}\frac{\delta (\sqrt{-g}R)}{\delta S_{kj}{}^l}\right) \right] \\
&= \kappa \sqrt{-g} \left[ \Sigma^{ij} + \frac{g^{li}}{2\kappa\sqrt{-g}} \stackrel{*}{\nabla}_k \left(\frac{\delta (\sqrt{-g}R)}{\delta S_{kj}{}^l}\right) \right] \\
&= \kappa \sqrt{-g} \Sigma^{ij} + \frac{g^{li}}{2} \stackrel{*}{\nabla}_k \left(\frac{\delta (\sqrt{-g}R)}{\delta S_{kj}{}^l}\right)
\end{align*}
After some tedious calculations, the geometrical terms simplify drastically to $G^{ij}$:
\[ - \frac{1}{\sqrt{-g}} \left[\frac{\delta(\sqrt{-g}R)}{\delta g_{ij}} + \frac{g^{li}}{2} \stackrel{*}{\nabla}_k \left(\frac{\delta (\sqrt{-g}R)}{\delta S_{kj}{}^l}\right)\right] =  \kappa \Sigma^{ij}
\quad\Rightarrow\quad G^{ij} = \kappa \Sigma^{ij} \]
Equation (2) can be written in terms of $\tau^{ijk}$ by an antisymmetrization:
\[ -\frac{g^{l[j}}{2\sqrt{-g}}\frac{\delta(\sqrt{-g}R)}{\delta S_{i]k}{}^l} = \kappa \mu^{[ji]k} = \kappa \tau^{ijk} \]
The left-hand side evaluates to modified torsion $T^{ijk}$, defined as torsion plus its trace:
\[ T_{ij}{}^k := S_{ij}{}^k + 2\delta^k_{[i}S_{j]l}{}^l\]
Thus we have obtained the two (sets of) field equations of Einstein-Cartan theory:
\begin{numcases}{}
G^{ij} = \kappa \Sigma^{ij}\\
T^{ijk} = \kappa \tau^{ijk}
\end{numcases}

\section{Physical Implications}
\subsection{Spin-torsion coupling}
The first set of field equations simply reduces to Einstein's equations in the case of zero spin. The second equation, however, reveals the coupling of torsion to the spin angular momentum of matter field. What is significant about (4), apart from the fact that it takes on a form analagous to (3), is that it is a set of simple algebraic equations rather than differential equations, which implies that torsion can only be zero outside of matter distribution and cannot propagate as a wave through vacuum -- it is bound to matter.
\subsection{Comparison to general relativity}
Furthermore, the fact that equation (4) is algebraic makes it possible to pull out the non-Riemannian part of $G^{ij}$ in equation (3) and substitute all $T^{ijk}$ with $\kappa\tau^{ijk}$, resulting in a combined field equation that involves only the Riemannian part of $G^{ij}$\cite{hehl1976general}:
\[ G^{ij}_{\{\}}  = \kappa \Bigg[\sigma^{ij} + \kappa\Big( -4\tau^{ik}{}_{[l}\tau^{jl}{}_{k]} 
-2\tau^{ikl}\tau^j{}_{kl} + \tau^{kli}\tau_{kl}{}^j + \frac{1}{2}g^{ij}\left( 4\tau_m{}^k{}_{[l}\tau^{ml}{}_{k]} + \tau^{mkl}\tau_{mkl} \right)\Big) \Bigg] \]
Compared to Einstein's field equations, which under this notation is
\[G^{ij}_{\{\}} = \kappa \sigma^{ij} \]
the term consisting of various products of the spin tensor $\tau$ manifests itself as the correction term, which obviously is very small due to the additional $\kappa$ factor. Suppose the matter distribution is made up of polarized elementary particles with particle mass $m$, spin $\hbar/2$, and number density $n$. From the combined field equation it is obvious that the mass density $\rho = mn$ receives a correction from spin of order $\kappa s^2 = \kappa (n\hbar/2)^2$. For spin to have an effect as significant as that of mass,
\[ mn \sim \kappa n^2\hbar^2 \quad\Rightarrow\quad n \sim \frac{m}{\kappa \hbar^2} \]
In terms of mass density\cite{hehl1976general},
\[ \rho_{critical} \sim \frac{m^2}{\kappa \hbar^2} \approx \begin{cases}
10^{47} \, \si{g.cm^{-3}} \text{ (electrons)} \\
10^{54} \,\si{g.cm^{-3}} \text{ (neutrons)}
\end{cases}\]
Thus, Einstein-Cartan theory only produces non-negligible deviations from general relativity under extremely high densities that might in fact be unphysical. 

\bibliography{ref}{}
\bibliographystyle{plainurl}
\end{document}