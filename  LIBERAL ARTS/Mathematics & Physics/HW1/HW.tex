\documentclass{article}
\usepackage[margin=1.2in]{geometry}
\usepackage{undertilde, amsmath, cancel}
\DeclareMathOperator{\tr}{tr}
\title{PHYS 7125 Homework 2}
\author{Wenqi He}
\begin{document}
\maketitle
\section{}
The local flatness property states that for each point $p$ on the manifold there exists a change of coordinates such that the metric $g_{\mu\nu}$ can be transformed into a $g_{\mu'\nu'}$ that satisfies: (i) $g_{\mu'\nu'} = \eta_{\mu'\nu'}$ and (ii) $g_{\mu'\nu',\sigma} = 0$ at point $p$. This can be shown by a Taylor expansion of $g_{\mu'\nu'}$ to the first order:
\[ g_{\mu'\nu'} = \frac{\partial x^\mu}{\partial x^{\mu'}}\frac{\partial x^\nu}{\partial x^{\nu'}} g_{\mu\nu} \]
\[ = \Big(x^\mu_{,\mu'} x^\nu_{,\nu'} g_{\mu\nu}\Big)\Big\rvert_p
	+ \Bigg( x^\mu_{,\mu'\lambda} x^\nu_{,\nu'} g_{\mu\nu}
	+ x^\mu_{,\mu'}x^\nu_{,\nu'\lambda}g_{\mu\nu}
	+ x^\mu_{,\mu'}x^\nu_{,\nu'} g_{\mu\nu,\lambda}\Bigg)\Big\rvert_p \epsilon + O(\epsilon^2) \]
The requirement is that 
\[ \Big(x^\mu_{,\mu'} x^\nu_{,\nu'} g_{\mu\nu}\Big)\Big\rvert_p = \eta_{\mu'\nu'} \]
\[ \Bigg( x^\mu_{,\mu'\lambda} x^\nu_{,\nu'} g_{\mu\nu}
	+ x^\mu_{,\mu'}x^\nu_{,\nu'\lambda}g_{\mu\nu}
	+ x^\mu_{,\mu'}x^\nu_{,\nu'} g_{\mu\nu,\lambda}\Bigg)\Big\rvert_p = 0 \]
The first equation has $16$ variables in $\partial x^\mu/\partial x^{\mu'}$ and $10$ equations, one for each indepedent entry of the metric, and the remaining 6 degrees of freedom exactly matches the dimension of the Lorentz group, under which the metric is preserved. Now that $\partial x^\mu/\partial x^{\mu'}$ is determined, the second equation will only have $4 \cdot 10 = 40$ variables in $\partial^2x^\mu/\partial x^{\mu'}\partial x^\lambda$ (partial derivatives commute) and coincidentally $10 \cdot 4 = 40$ equations corresponding to the entries of $g_{\mu\nu,\lambda}$ (metric is symmetric by definition), so the system is uniquely determined, which proves that such transformation always exists.

\section{}
\subsection*{a}
\begin{align*}
g^{\alpha\beta}{}_{,\gamma} &=\Big( g^{\alpha\nu}g^{\beta\mu}g_{\mu\nu}\Big)_{,\gamma} \\
&= g^{\alpha\nu}{}_{,\gamma}g^{\beta\mu}g_{\mu\nu} + g^{\alpha\nu}g^{\beta\mu}{}_{,\gamma}g_{\mu\nu} + g^{\alpha\nu}g^{\beta\mu}g_{\mu\nu}{}_{,\gamma} \\
&= 2g^{\alpha\nu}g^{\beta\mu}{}_{,\gamma}g_{\mu\nu} + g^{\alpha\nu}g^{\beta\mu}g_{\mu\nu}{}_{,\gamma} \\
&= 2g^{\alpha\nu}\Big( g^{\beta\mu}{}_{,\gamma}g_{\mu\nu} + g^{\beta\mu}g_{\mu\nu}{}_{,\gamma}\Big) - g^{\alpha\nu}g^{\beta\mu}g_{\mu\nu}{}_{,\gamma} \\
&= 2g^{\alpha\nu}\Big(g^{\beta\mu}g_{\mu\nu}\Big)_{,\gamma} - g^{\alpha\nu}g^{\beta\mu}g_{\mu\nu}{}_{,\gamma} \\
&= \cancel{2g^{\alpha\nu}\delta^\beta_{\nu,\gamma}} - g^{\alpha\nu}g^{\beta\mu}g_{\mu\nu}{}_{,\gamma} 
= - g^{\alpha\nu}g^{\beta\mu}g_{\mu\nu}{}_{,\gamma} 
\end{align*}
\subsection*{b}
From the two identities we can derive the formula:
\begin{align*} \frac{d}{d\epsilon}\det(A) &=  \lim_{\epsilon \rightarrow 0} \frac{\det(A + \epsilon \frac{d}{d\epsilon}A + O(\epsilon^2)) - \det(A)}{\epsilon}\\
	&= \lim_{\epsilon \rightarrow 0} \frac{\det(A(I + \epsilon A^{-1}\frac{d}{d\epsilon}A)) - \det(A)}{\epsilon} \\
	&=  \lim_{\epsilon \rightarrow 0} \frac{\det(A)\det(I + \epsilon A^{-1}\frac{d}{d\epsilon}A) - \det(A)}{\epsilon} \\
	&=  \det(A) \lim_{\epsilon \rightarrow 0} \frac{\det(I + \epsilon A^{-1}\frac{d}{d\epsilon}A) - 1}{\epsilon} \\
	&=  \det(A) \lim_{\epsilon \rightarrow 0} \frac{1 + \epsilon \tr(A^{-1}\frac{d}{d\epsilon}A) + O(\epsilon^2) - 1}{\epsilon} \\
	&=  \det(A) \tr(A^{-1}\frac{d}{d\epsilon}A) \\
\end{align*}
Apply the formula to $g_{\mu\nu}$, replacing $d/d\epsilon$ with $\partial_\alpha$
\[ g_{,\alpha} = g \cdot \tr(g^{\sigma\mu}g_{\mu\nu,\alpha}) = g g^{\nu\mu}g_{\mu\nu,\alpha} \]
\subsection*{c}
From right to left
\begin{align*}
&\indent -(-g)^{-1/2}\Big[g^{\alpha\beta}(-g)^{1/2}\Big]_{,\beta} \\
&= -(-g)^{-1/2}\Big[g^{\alpha\beta}{}_{,\beta}(-g)^{1/2} + g^{\alpha\beta}(-g)^{1/2}_{,\beta} \Big]\\
&= -(-g)^{-1/2}\Big[g^{\alpha\beta}{}_{,\beta}(-g)^{1/2} - \frac{1}{2}g^{\alpha\beta}(-g)^{-1/2}g_{,\beta} \Big]\\
&= - g^{\alpha\beta}{}_{,\beta} + \frac{1}{2}g^{\alpha\beta}(-g)^{-1}g_{,\beta}\\
&= g^{\mu\beta}g^{\nu\alpha}g_{\mu\nu,\beta} + \frac{1}{2}g^{\alpha\beta}(-g)^{-1}gg^{\mu\nu}g_{\mu\nu,\beta}\\
&= g^{\mu\beta}g^{\nu\alpha}g_{\mu\nu,\beta} - \frac{1}{2}g^{\alpha\beta}g^{\mu\nu}g_{\mu\nu,\beta}\\
&= \frac{1}{2}\Big(g^{\mu\beta}g^{\nu\alpha}g_{\mu\nu,\beta}  + g^{\mu\beta}g^{\nu\alpha}g_{\mu\nu,\beta} - g^{\alpha\beta}g^{\mu\nu}g_{\mu\nu,\beta} \Big) \\
&= \frac{1}{2}\Big(g^{\mu\nu}g^{\beta\alpha}g_{\mu\beta, \nu}  + g^{\nu\beta}g^{\mu\alpha}g_{\nu\mu,\beta} - g^{\alpha\beta}g^{\mu\nu}g_{\mu\nu,\beta} \Big) \\
&= \frac{1}{2}\Big(g^{\mu\nu}g^{\beta\alpha}g_{\mu\beta, \nu}  + g^{\nu\mu}g^{\beta\alpha}g_{\nu\beta,\mu} - g^{\alpha\beta}g^{\mu\nu}g_{\mu\nu,\beta} \Big) \\
&= g^{\mu\nu}\cdot \frac{1}{2}g^{\alpha\beta} \Big(g_{\beta\mu, \nu}  +g_{\beta\nu,\mu} - g_{\mu\nu,\beta} \Big)
= g^{\mu\nu} \Gamma^\alpha_{\mu\nu}
\end{align*}
\subsection*{d}
\begin{align*}
LHS &= A^\alpha{}_{,\alpha} + \Gamma^{\alpha}_{\alpha\lambda}A^\lambda \\
&= A^\alpha{}_{,\alpha} + \frac{1}{2}g^{\alpha\beta}\Big(g_{\beta\alpha,\lambda} + g_{\beta\lambda, \alpha} - g_{\alpha\lambda,\beta}\Big)A^\lambda \\
&= A^\alpha{}_{,\alpha} + \frac{1}{2}\Big(g^{\alpha\beta}g_{\beta\alpha,\lambda} + g^{\alpha\beta}g_{\beta\lambda, \alpha} -  g^{\alpha\beta}g_{\alpha\lambda,\beta}\Big)A^\lambda \\
&= A^\alpha{}_{,\alpha} + \frac{1}{2}\Big(g^{\alpha\beta}g_{\beta\alpha,\lambda} + \cancel{g^{\alpha\beta}g_{\beta\lambda, \alpha}} -  \cancel{g^{\beta\alpha}g_{\beta\lambda,\alpha}} \Big)A^\lambda  \\
&= A^\alpha{}_{,\alpha} + \frac{1}{2}g^{\alpha\beta}g_{\beta\alpha,\lambda}A^\lambda \\
&= A^\alpha{}_{,\alpha} + \frac{1}{2}g^{\mu\nu}g_{\mu\nu,\alpha}A^\alpha \\\\
RHS &= (-g)^{-1/2}\Big[(-g)^{1/2}A^\alpha\Big]_{,\alpha} \\
&= (-g)^{-1/2}\Big[(-g)^{1/2}A^\alpha{}_{,\alpha} + (-g)^{1/2}_{,\alpha}A^\alpha \Big] \\
& = A^\alpha{}_{,\alpha} - \frac{1}{2}(-g)^{-1/2} (-g)^{-1/2}g_{,\alpha}A^\alpha \\ 
& = A^\alpha{}_{,\alpha} + \frac{1}{2}(g)^{-1}gg^{\mu\nu}g_{\mu\nu,\alpha} A^\alpha \\ 
& = A^\alpha{}_{,\alpha} + \frac{1}{2}g^{\mu\nu}g_{\mu\nu,\alpha} A^\alpha = LHS \\ 
\end{align*}
\subsection*{e}
Since $g = \det(g_{\mu\nu}) = \prod \lambda_i = -1 \cdot 1 \cdot 1 \cdot 1 = -1$ is constant,
\begin{align*}
\epsilon_{\alpha\beta\gamma\delta;\mu} = \Big((-g)^{1/2}\tilde{\epsilon}_{\alpha\beta\gamma\delta}\Big)_{;\mu}
= (-g)^{1/2}_{;u}\tilde{\epsilon}_{\alpha\beta\gamma\delta} = 0
\end{align*}
\section{}
\subsection*{a}
Since $u_\alpha u^\alpha  = -1$,
\[
	P_{\alpha\beta}v^\beta u^\alpha = g_{\alpha\beta} v^\beta u^\alpha +  u_{\alpha}u_{\beta} v^\beta u^\alpha
	= v_\alpha u^\alpha - u_\beta v^\beta = 0
\]
\subsection*{b}
From (a), $u^\beta v_{\perp_\beta} = u_{\beta} v_{\perp}^\beta = 0$, therefore
\[
	P_{\alpha\beta}v_{\perp}^\beta =  g_{\alpha\beta} v_{\perp}^\beta +  u_{\alpha}u_{\beta} v_{\perp}^\beta
	=  g_{\alpha\beta} v_{\perp}^\beta +  0 = v_{\perp\alpha}
\]
\subsection*{c}
\[
	P_{\alpha\beta} := g_{\alpha\beta}  - ({q_\lambda q^\lambda})^{-1} q_{\alpha}q_{\beta}
\]
\textit{Proof:} Carrying out the same calculation as above,
\[  P_{\alpha\beta}v^\beta q^\alpha = g_{\alpha\beta} v^\beta q^\alpha - ({q_\lambda q^\lambda})^{-1} q_{\alpha}q_{\beta} v^\beta q^\alpha = v_\alpha q^\alpha - q_\beta v^\beta = 0 \]
\[
	P_{\alpha\beta}v_{\perp}^\beta =  g_{\alpha\beta} v_{\perp}^\beta  - ({q_\lambda q^\lambda})^{-1} q_{\alpha}q_{\beta}  v_{\perp}^\beta
	=  g_{\alpha\beta} v_{\perp}^\beta - 0 = v_{\perp\alpha}
\]
\subsection*{d}
The candidates for the projection tensor should take the form $Ag_{\alpha\beta} + Bk_\alpha k_\beta$. In order for the projection to be orthogonal,
\[  \Big(Ag_{\alpha\beta} + Bk_\alpha k_\beta\Big)v^\beta k^\alpha = Av_\alpha k^\alpha + 0 = 0 \]
Since $v_\alpha k^\alpha \neq 0$ in general, $A$ must be zero. However, in order that $P_{\alpha\beta}v_\perp^\beta = v_{\perp\alpha}$,
\[ \Big(Ag_{\alpha\beta} + Bk_\alpha k_\beta\Big)v_\perp^\beta = Av_{\perp\alpha} + 0 = v_{\perp\alpha} \]
$A$ must equal $1$, therefore there is no unique projection tensor, which must satisfy both conditions.

\section{}
\subsection*{a}
\begin{align*}
 \nabla_{\utilde{u}}w_\mu &= \nabla_{\frac{d}{d\tau}}w_\mu = \nabla_{\frac{dx^\alpha}{d\tau}\partial_\alpha}w_\mu
= \frac{dx^\alpha}{d\tau} \nabla_{\partial_\alpha}w_\mu
= \frac{dx^\alpha}{d\tau} \Big(\partial_\alpha w_\mu - \Gamma^{\beta}_{\alpha\mu}w_\beta \Big) \\
&= \frac{dw_\mu}{d\tau} - \Gamma^\beta_{\alpha\mu}u^\alpha w_\beta = 0
\end{align*}
\subsection*{b}
Since $u_\mu u^\mu \equiv -1$, it must be true that $\nabla_{\utilde{u}} \Big(u_\mu u^\mu\Big) \equiv 0$. Using the answer above, indeed
\begin{align*}
 \nabla_{\utilde{u}} \Big(u_\mu u^\mu\Big) 
&= u_\mu \nabla_{\utilde{u}} u^\mu + u^\mu \nabla_{\utilde{u}} u_\mu 
=  u_\mu \Big(\frac{du^\mu}{d\tau} + \Gamma^\mu_{\alpha\beta}u^\alpha u^\beta \Big)
	+ u^\mu \Big(  \frac{du_\mu}{d\tau} - \Gamma^\beta_{\alpha\mu}u^\alpha u_\beta \Big)\\
&=  u_\mu\frac{du^\mu}{d\tau} + \cancel{\Gamma^\mu_{\alpha\beta}u_\mu u^\alpha u^\beta}
	+  u^\mu  \frac{du_\mu}{d\tau} - \cancel{\Gamma^\beta_{\alpha\mu}u^\mu  u^\alpha u_\beta}\\
&= \frac{d}{d\tau}\Big(u_\mu u^\mu \Big) = 0
\end{align*}
\subsection*{c}
Suppose $\lambda$ is an affine parameter for a null-geodesic, and $\sigma$ non-affine:
\begin{align*}
 \nabla_{\utilde{u}}u^\mu &= \nabla_{\frac{d}{d\sigma}}\frac{dx^\mu}{d\sigma} 
	= \nabla_{\frac{d\lambda}{d\sigma}\frac{d}{d\lambda}}\Big(\frac{d\lambda}{d\sigma}\frac{dx^\mu}{d\lambda}\Big)
	= \frac{d\lambda}{d\sigma} \nabla_{\frac{d}{d\lambda}}\Big(\frac{d\lambda}{d\sigma}\frac{dx^\mu}{d\lambda}\Big) \\
	&= \frac{d\lambda}{d\sigma}\frac{d\lambda}{d\sigma} \cancel{\nabla_{\frac{d}{d\lambda}}\frac{dx^\mu}{d\lambda}}
	+ \frac{d\lambda}{d\sigma} \frac{dx^\mu}{d\lambda}\frac{d}{d\lambda}\frac{d\lambda}{d\sigma} \\
	&= \frac{d}{d\lambda}\frac{d\lambda}{d\sigma} u^\mu =: - \kappa u^\mu
\end{align*}
where $\nabla_{\frac{d}{d\lambda}}\frac{dx^\mu}{d\lambda} = 0$ by the definition of affineness.
\end{document}