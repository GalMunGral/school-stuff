\documentclass{article}
\usepackage[margin=1.2in]{geometry}
\usepackage{undertilde, amsmath}
\title{PHYS 7125 Homework 1}
\author{Wenqi He}
\begin{document}
\maketitle
\section{}
\subsection*{a}
If $\Delta s^2 = 0$ then $\Delta t = \sqrt{\Delta x^2 + \Delta y^2 + \Delta z^2}$. A particle that travels $\Delta x^\alpha$ travels at the speed of light. And since the speed of light is constant in all reference frames, in another coordinate system $x^{\alpha'}$, it still travels at the speed of light, that is, $\Delta t' = \sqrt{\Delta x'^2 + \Delta y'^2 + \Delta z'^2}$. In other words, $\Delta s'^2 = 0$.
\subsection*{b}
Expressing $Q$ in terms of $\Delta x^\alpha$ gives another quadratic form:
\[ Q = \eta_{\alpha'\beta'}\Delta x^{\alpha'} \Delta x^{\beta'}  
	= \eta_{\alpha'\beta'} \Lambda^{\alpha'}{}_\alpha \Delta x^\alpha  \Lambda^{\beta'}{}_\beta \Delta x^\beta
	= \Big(\Lambda^{\alpha'}{}_\alpha\Lambda^{\beta'}{}_\beta\eta_{\alpha'\beta'}\Big) \Delta x^\alpha \Delta x^\beta
	= \phi_{\alpha\beta}\Delta x^\alpha \Delta x^\beta \]
On the light cone $\Delta s^2 = 0$. And from the result of (a), $Q = \Delta s'^2 = \Delta s^2 = 0$
\subsection*{c}
On the intersection $Q = 0$ everywhere. By spherical symmetry in the spatial components of $\Delta x^\alpha$, $Q$ must be invariant under a reverse of sign of any spatial coordinate, therefore all cross terms are eliminated. $Q$ must also be invariant under a permutation of spatial coordinates, so the remaining $\Delta x^2, \Delta y^2, \Delta z^2$ must have the same coefficient. Therefore the general form is 
\[ Q = c_1\Delta t^2 + c_2(\Delta x^2 + \Delta y^2 + \Delta z^2) \]
\subsection*{d}
On the intersection, which lies on the light cone, $\Delta t^2 = \Delta x^2$,
\[ Q = c_1\Delta t^2 + c_2 \Delta x^2 = (c_1  + c_2) \Delta t^2 = 0 \quad\Rightarrow\quad c_1 = -c_2\]
\[ Q = c_2(-\Delta t ^2 +\Delta x^2 + \Delta y^2 + \Delta z^2) = c_2\eta_{\alpha\beta}\Delta x^\alpha \Delta x^\beta \]
\subsection*{e}
The constant $c_2$ applies to transformations between any two coordinate systems, including the trivial one between one coordinate system and itself, therefore $c_2$ must equal $1$.

\section{}
\subsection*{a}
Rename the dummy indices, and apply the definition of symmetric/antisymmetric tensors:
\[ A_{\mu\nu}S^{\mu\nu} = A_{\nu\mu}S^{\nu\mu} = (-A_{\mu\nu})S^{\mu\nu}
	\quad\Rightarrow\quad A_{\mu\nu}S^{\mu\nu} = 0 \]
\subsection*{b}
Using the same trick as above,
\[ V^{\nu\mu}A_{\mu\nu} = V^{\mu\nu}A_{\nu\mu} = - V^{\mu\nu}A_{\mu\nu}
	\quad\Rightarrow\quad \frac{1}{2}\Big( V^{\mu\nu}A_{\mu\nu} - V^{\nu\mu}A_{\mu\nu}\Big) =  V^{\mu\mu}A_{\mu\nu}  \]
\[ V^{\nu\mu}S_{\mu\nu} = V^{\mu\nu}S_{\nu\mu} = V^{\mu\nu}S_{\mu\nu}
	\quad\Rightarrow\quad \frac{1}{2}\Big( V^{\mu\nu}S_{\mu\nu} + V^{\nu\mu}S_{\mu\nu}\Big) =  V^{\mu\mu}S_{\mu\nu}  \]
\subsection*{c}
When acting on (co)vectors, the tensor produces a scalar, which is invariant under transformations: 
\begin{align*}
	 T^{\alpha'}{}_{\beta'}{}^{\gamma'}u_{\alpha'}v^{\beta'}w_{\gamma'} 
	&= T^{\alpha}{}_{\beta}{}^{\gamma}u_{\alpha}v^{\beta}w_{\gamma}\\
	&= T^{\alpha}{}_{\beta}{}^{\gamma}\Lambda^{\alpha'}{}_{\alpha} u_{\alpha'}\Lambda^{\beta}{}_{\beta'}v^{\beta'} \Lambda^{\gamma'}{}_{\gamma}w_{\gamma} \\
	&= \Big(\Lambda^{\alpha'}{}_{\alpha} \Lambda^{\beta}{}_{\beta'} \Lambda^{\gamma'}{}_{\gamma} T^{\alpha}{}_{\beta}{}^{\gamma}\Big) u_{\alpha'}v^{\beta'} w_{\gamma}
\end{align*}
Since this holds true for any (co)vectors, $T^{\alpha'}{}_{\beta'}{}^{\gamma'}= \Lambda^{\alpha'}{}_{\alpha} \Lambda^{\beta}{}_{\beta'} \Lambda^{\gamma'}{}_{\gamma} T^{\alpha}{}_{\beta}{}^{\gamma}$.
\subsection*{d}
\[g_{\alpha\beta}g^{\beta\sigma}g^{\alpha\gamma} = \delta_{\alpha}^\sigma g^{\alpha\gamma} = g^{\sigma\gamma}\]
\[g_{\sigma\beta} g_{\gamma\alpha} g^{\alpha\beta} = g_{\sigma\beta} \delta_{\gamma}^\beta  = g_{\sigma\gamma}\]
\[g^\alpha{}_\beta = g^{\alpha\sigma} g_{\sigma\beta} = g_{\beta\sigma}g^{\sigma\alpha} = \delta_\beta^\alpha \]

\section{}
\subsection*{a}
\[
	X^{\mu}{}_{\nu} = X^{\mu\gamma}{g}_{\gamma\nu}
	= \begin{pmatrix}
		2 & 0 & 1 & -1 \\
		-1 & 0 & 3 & 2 \\
		-1 & 1 & 0 & 0 \\
		-2 & 1 & 1 & -2
	\end{pmatrix}
	\begin{pmatrix}
		-1 & 0 & 0 & 0 \\
		0 & 1 & 0 & 0 \\
		0 & 0 & 1 & 0 \\
		0 & 0 & 0 & 1
	\end{pmatrix}
	= \begin{pmatrix}
		-2 & 0 & 1 & -1 \\
		1 & 0 & 3 & 2 \\
		1 & 1 & 0 & 0 \\
		2 & 1 & 1 & -2
	\end{pmatrix}
\]

\subsection*{b}
\[
	X_{\mu}{}^{\nu} = g_{\mu\gamma}X^{\gamma\nu}
	= \begin{pmatrix}
		-1 & 0 & 0 & 0 \\
		0 & 1 & 0 & 0 \\
		0 & 0 & 1 & 0 \\
		0 & 0 & 0 & 1
	\end{pmatrix}
	\begin{pmatrix}
		2 & 0 & 1 & -1 \\
		-1 & 0 & 3 & 2 \\
		-1 & 1 & 0 & 0 \\
		-2 & 1 & 1 & -2
	\end{pmatrix}
	= \begin{pmatrix}
		-2 & 0 & -1 & 1 \\
		-1 & 0 & 3 & 2 \\
		-1 & 1 & 0 & 0 \\
		-2 & 1 & 1 & -2
	\end{pmatrix}
\]
\subsection*{c}
\[
	X^{(\mu\nu)} = \frac{1}{2}\Big( X^{\mu\nu} + X^{\nu\mu} \Big)
	= \begin{pmatrix}
		2 & -1/2 & 0 & -3/2 \\
		-1/2 & 0 & 2 & 3/2 \\
		0 & 2 & 0 & 1/2 \\
		-3/2 & 3/2 & 1/2 & -2
	\end{pmatrix}
\]
\subsection*{d}
\[
	X_{\mu\nu} = X_{\mu}{}^{\gamma}g_{\gamma\nu}
	= \begin{pmatrix}
		-2 & 0 & -1 & 1 \\
		-1 & 0 & 3 & 2 \\
		-1 & 1 & 0 & 0 \\
		-2 & 1 & 1 & -2
	\end{pmatrix}
	\begin{pmatrix}
		-1 & 0 & 0 & 0 \\
		0 & 1 & 0 & 0 \\
		0 & 0 & 1 & 0 \\
		0 & 0 & 0 & 1
	\end{pmatrix}
	=  \begin{pmatrix}
		2 & 0 & -1 & 1 \\
		1 & 0 & 3 & 2 \\
		1 & 1 & 0 & 0 \\
		2 & 1 & 1 & -2
	\end{pmatrix}
\]
\[
	X_{[\mu\nu]} = \frac{1}{2} \Big( X_{\mu\nu} - X_{\nu\mu} \Big)
 	= \begin{pmatrix}
		0 & -1/2 & -1 & -1/2 \\
		1/2 & 0 & 1 & 1/2 \\
		1 & -1 & 0 & -1/2 \\
		1/2 & -1/2 & 1/2 & 0
	\end{pmatrix}
\]
\subsection*{e}
\[ X^\lambda{}_\lambda = -2 + 0 + 0  -2 = -4 \]
\subsection*{f}
\[ v^\mu v_\mu = g_{\mu\nu} v^\mu v^\nu = -(-1)^2 + 2^2 + 0^2 + (-2)^2 = 7  \]
\subsection*{g}
\[
	v_\mu = g_{\mu\nu} v^\nu
	= \begin{pmatrix}
		-1 & 0 & 0 & 0 \\
		0 & 1 & 0 & 0 \\
		0 & 0 & 1 & 0 \\
		0 & 0 & 0 & 1
	\end{pmatrix}
	\begin{pmatrix}
		-1 \\ 2 \\ 0 \\ -2
	\end{pmatrix}
	= 	\begin{pmatrix}
		1 & 2 & 0 & -2
	\end{pmatrix}
\]
\[
	v_\mu X^{\mu\nu}
	= 	\begin{pmatrix}
		1 & 2 & 0 & -2
	\end{pmatrix}
	\begin{pmatrix}
		2 & 0 & 1 & -1 \\
		-1 & 0 & 3 & 2 \\
		-1 & 1 & 0 & 0 \\
		-2 & 1 & 1 & -2
	\end{pmatrix}
	= \begin{pmatrix}
		4 \\ -2 \\ 5 \\ 7
	\end{pmatrix}
\]




































\end{document}